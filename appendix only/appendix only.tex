%%%%%%%%%%%%%%%%%%%%%%%%%%%%%%%%%%%%%%%%%%%%%%%%%%%
% The following is the Latex source file of the   %
%%%%%%%%%%%%%%%%%%%%%%%%%%%%%%%%%%%%%%%%%%%%%%%%%%%
%
%\documentclass[aps,letterpaper,onecolumn,preprintnumbers,floatfix,superscriptaddress,11pt]{revtex4-1}
%\documentclass[aps,prl,letterpaper,showpacs,twocolumn,preprintnumbers,floatfix,superscriptaddress]{revtex4-1}
%\documentclass[aps,letterpaper,twocolumn,preprintnumbers,floatfix,superscriptaddress]{revtex4}
%\documentclass[preprint,showpacs,preprintnumbers,amsmath,amssymb]{revtex4}
%\documentclass[preprint,aps]{revtex4}
%\documentclass[preprint,aps,draft]{revtex4}
%\documentclass[prb]{revtex4}% Physical Review D
%\documentclass[12pt]{article}
\documentclass[a4paper,11pt]{article}
%\pdfoutput=1 % if your are submitting a pdflatex (i.e. if you have
             % images in pdf, png or jpg format)

\usepackage{jheppub} % for details on the use of the package, please
                     % see the JHEP-author-manual

%\usepackage[T1]{fontenc} % if needed

%\documentclass[a4paper,11pt]{article}
%\pdfoutput=1 % if your are submitting a pdflatex (i.e. if you have
             % images in pdf, png or jpg format)

%\usepackage{jheppub}
\usepackage[T1]{fontenc}
\usepackage{multirow}
\usepackage{longtable}
%\usepackage{dcolumn}
\usepackage{graphicx}% Include figure files
\usepackage{dcolumn}% Align table columns on decimal point
\usepackage{bm}% bold math
%\usepackage{epsfig,cancel}
\usepackage{epsfig}
\usepackage{gensymb}
\usepackage{mathrsfs}
\usepackage{extpfeil}
\usepackage{extarrows}
\usepackage{float}
\usepackage{pdfpages}
%\documentstyle[epsf,aps]{revtex}
%\documentclass[12pt]{article}
%\usepackage{epsfig}
\usepackage{amsmath}
\usepackage{amssymb}
\usepackage{graphicx,bm}
\usepackage{natbib}
\usepackage{slashed}
\usepackage[makeroom]{cancel}
\usepackage{dsfont}
\usepackage{longtable}
\usepackage{multirow}
\usepackage{youngtab}
\usepackage{young}
%\usepackage{ytableau}
\usepackage{tikz}
\usepackage{cleveref}

\newcommand{\fd}{D}
\newcommand{\pd}{d}
\newcommand{\CF}{{\{ \mathcal{C}.F \}}}
\newcommand{\BH}{{\{\mathcal{B} \cdot H \}}}
\newcommand{\lsbk}[1]{\left[{#1}\right|}
\newcommand{\labk}[1]{\left\langle{#1}\right|}
\newcommand{\rsbk}[1]{\left|{#1}\right]}
\newcommand{\rabk}[1]{\left|{#1}\right\rangle}
\newcommand{\sbk}[1]{\left[{#1}\right]}
\newcommand{\abk}[1]{\left\langle{#1}\right\rangle}
%\usepackage[utf8]{inputenc}
%\usepackage{subcaption}
%\usepackage{axodraw}
%\usepackage{hyperref}
%\usepackage[mathlines]{lineno}% Enable numbering of text and display math
%\usepackage{color}
%\usepackage{mathrsfs}
 \usepackage[titletoc]{appendix}
% \usepackage[toc,page]{appendix}
\def\la{\mathrel{\mathpalette\fun <}}
\def\ga{\mathrel{\mathpalette\fun >}}
\def\fun#1#2{\lower3.6pt\vbox{\baselineskip0pt\lineskip.9pt
  \ialign{$\mathsurround=0pt#1\hfil##\hfil$\crcr#2\crcr\sim\crcr}}}

\usepackage{tikz}
\usetikzlibrary{decorations.pathmorphing,decorations.markings,trees,shapes}

\newcommand{\CPV}{CP\!\!\!\!\!\!\!\!\raisebox{0pt}{\small$\diagup$}}
\newcommand{\half}{{\textstyle\frac{1}{2}}}
\newcommand{\third}{{\textstyle\frac{1}{3}}}
\newcommand{\threehalf}{{\textstyle\frac{3}{2}}}
\newcommand{\threefourth}{{\textstyle\frac{3}{4}}}
\newcommand{\fourth}{{\textstyle\frac{1}{4}}}
\newcommand{\Tr}{{\rm Tr\hskip 2pt}}
\def\lsim{\mathrel{\rlap{\raise 2.5pt \hbox{$<$}}\lower 2.5pt\hbox{$\sim$}}}
\def\gsim{\mathrel{\rlap{\raise 2.5pt \hbox{$>$}}\lower 2.5pt\hbox{$\sim$}}}

\renewcommand{\Re}{{\rm Re\thinspace}}
\renewcommand{\Im}{{\rm Im\thinspace}}
\newcommand{\op}{\mathcal{O}}

\def\f{kinematic factor}

\def\btt#1{{\tt$\backslash$#1}}
\input epsf
\def\plotlittle#1{\centering \leavevmode
\epsfxsize= 0.8\columnwidth \epsfbox{#1}}
\def\plotone#1{\centering \leavevmode
\epsfxsize= 1.0\columnwidth \epsfbox{#1}}
\def\plottwo#1{\centering \leavevmode
\epsfxsize= 1.0\columnwidth \epsfbox{#1}}
\def\plotfiddle#1#2#3#4#5#6#7{\centering \leavevmode
\vbox to#2{\rule{0pt}{#2}} \special{psfile=#1 voffset=#7
hoffset=#6 vscale=#5 hscale=#4 angle=#3}}
%\def\plotrotate#1{\centering \leavevmode
%\epsfxsize= 0.8\columnwidth  \epsfbox{#1  {angle=-90}}}
\def\plotrotate#1{\centering
%\leavevmode
%\epsfxsize= 1.7\columnwidth
\epsfbox{#1 angle=-90}}

\newenvironment{tablehere}{\def\@captype{table}}{}
\newcommand{\tableskip}{\\[-6pt]}


\def\note#1{[{\bf {*** #1}}]}
\def\red#1{\textcolor{red}{#1}}
\newcommand{\blue}[1]{\textcolor{blue}{#1}}
%\def\blue#1{  \textcolor{blue}{#1}}
%\def\blue#1{{  \textcolor{blue}{{#1}}}}
\usepackage{color}



\newcommand{\iprime}{i^{\prime}}

\newcommand{\onec}{1^{c}}
\newcommand{\twoc}{2^{c}}
\newcommand{\threec}{3^{c}}
\newcommand{\fourc}{4^{c}}
\newcommand{\fivec}{5^{c}}
\newcommand{\sixc}{6^{c}}
\newcommand{\sevenc}{7^{c}}
\newcommand{\eightc}{8^{c}}
\newcommand{\ninec}{9^{c}}
\newcommand{\nuprime}{N_U^{\prime}}
\newcommand{\ndprime}{N_D^{\prime}}
\newcommand{\mathscrL}[0]{\mathscr{L}}
\newcommand{\mathscrl}[1]{\mathscrL_{\textrm{#1}}}
\newcommand{\sbeta}[0]{s_{\beta}}
\newcommand{\cbeta}[0]{c_{\beta}}
\newcommand{\rtilde}[0]{\tilde{R}}
\newcommand{\bbmixing}[0]{B_s^0 - \bar{B}_s^0 }
\newcommand{\comment}[1]{}

\newcommand{\be}{\begin{equation}}
\newcommand{\ee}{\end{equation}}
\newcommand{\bea}{\begin{eqnarray}}
\newcommand{\eea}{\end{eqnarray}}

% =============== MX Commands ==========================
\newcommand{\mc}[1]{\mathcal{#1}}
\newcommand{\bi}[1]{\textbf{\textit{#1}}}
\newcommand{\vev}[1]{\langle #1 \rangle}
\newcommand{\ket}[1]{| #1 \rangle}
\newcommand{\bra}[1]{\langle #1 |}
\newcommand{\antiket}[1]{| #1 ]}
\newcommand{\antibra}[1]{[ #1 |}
\newcommand{\ie}{{\textit{i.e.}}~}
\newcommand{\eg}{{\textit{e.g.}}~}
\newcommand{\cf}{{\textit{c.f.}}~}
\newcommand{\st}{{\textit{s.t.}}~}
\newcommand{\hc}{{\textit{h.c.}}~}
\newcommand{\cc}{{\textit{c.c.}}~}
\newcommand{\lrto}{\leftrightarrow}
\newcommand{\eq}[1]{\begin{equation}\begin{split} #1 \end{split}\end{equation}}
\newcommand{\eqs}[1]{\begin{align} #1 \end{align}}
\newcommand{\diff}[1]{{\rm d} #1 \,} % \diff{x} gives dx with correct font
\newcommand{\Diff}[2]{{\rm d}^{#2} #1 \,} % \Diff{x}{2} gives d^{2}x with correct font
\newcommand*\Dbox{\mathop{}\!\mathbin\Box}
\newcommand{\todo}[1]{{ \color{red}{#1}}\marginpar{\color{red}{$\Longleftarrow$}}}	% To Do Marks
% ======================================================

\renewcommand\arraystretch{2}
\makeatletter
\gdef\@fpheader{}
\makeatother

\begin{document}

\title{\boldmath Appendix of Operators for \\ The New Formulation of Higgs Effective Field Theory}

\author[a,b,c]{Zi-Yu Dong}
\author[d]{Teng Ma,}
\author[e,f]{Jing Shu}
\author[a,b]{Zi-Zheng Zhou}

% The "\note" macro will give a warning: "Ignoring empty anchor..."
% you can safely ignore it.

\affiliation[a]{CAS Key Laboratory of Theoretical Physics, Institute of Theoretical Physics,
Chinese Academy of Sciences, Beijing 100190, China.}
\affiliation[b]{School of Physical Sciences, University of Chinese Academy of Sciences, Beijing 100190, P. R. China.}
\affiliation[c]{Department of Physics, LEPP, Cornell University, Ithaca, NY 14853, USA}

\affiliation[d]{Physics Department, Technion -- Israel Institute of Technology, Haifa 3200003, Israel}

\affiliation[e]{School of Physics and State Key Laboratory of Nuclear Physics and Technology, Peking University, Beijing 100871, China}
\affiliation[f]{Center for High Energy Physics, Peking University, Beijing 100871, China}

%\affiliation[g]{School of Fundamental Physics and Mathematical Sciences, Hangzhou Institute for Advanced\\ Study, University of Chinese Academy of Sciences, Hangzhou 310024, China}
%\affiliation[h]{International Center for Theoretical Physics Asia-Pacific, Beijing/Hangzhou, China}

% e-mail addresses: one for each author, in the same order as the authors
\emailAdd{zd79@cornell.edu}
\emailAdd{t.ma@campus.technion.ac.il}
\emailAdd{jshu@pku.edu.cn}
\emailAdd{zhouzizheng@itp.ac.cn}


\abstract{
  This appendix is a supplementary material for arxiv:2211.16515. This appendix provides a complete listing of the possible operators for Standard Model particles.
}

\maketitle

\tableofcontents

\clearpage
\appendix

\section{The four-point HEFT operators up to dim-8}
\label{app:d8bases}
% \input{Define.tex}
% \includepdf[page={1-5}]{result_output.pdf}
In this appendix, we present the complete HEFT operators for some 4-point scattering processes involving SM particles up to dimension-8. In these bases, the Greek letters represent Lorentz indices while the English letters represent QCD color indices. 
% The definitions of notation can be found around Eq.~(\ref{eq:matching}) and (\ref{eq:uugg_oprator}), with $D_\nu$ representing the covariant derivative.


Note that sometimes different operators correspond to the same amplitude bases, and we indicate them in parentheses. To obtain them explicitly, we merely need to substitute the fields in the EFT operators as suggested in parentheses.
The replacement rules are shown in the table below, where particles enclosed in the same braces can be interchanged with each other.
\begin{table}[H]
    \begin{center}
    \begin{tabular}{l|l}
    \hline
     \textbf{Lorentz structure}
     & \textbf{Fields} \\ \hline
     \text{left handed spinor} & $\{u_L,\,d_L,\,\bar{u}_R,\,\bar{d}_R\},\ \{e_L,\,\nu_L,\,\bar{e}_R,\,\bar{\nu}_R\}$ \\ \hline
     \text{right handed spinor} & $\{u_R,\,d_R,\,\bar{u}_L,\,\bar{d}_L\},\ \{e_R,\,\nu_R,\,\bar{e}_L,\,\bar{\nu}_L\}$\\ \hline
     \text{vector} & $\{W^+,\,W^-\}$\\ \hline
    \end{tabular}
    \end{center}
\end{table}

The following are some four-point scattering that are common on colliders and have important detection significance in particle physics:
\begin{itemize}
    \item Drell-Yan process: The scattering process involving a quark-antiquark pair annihilating into a virtual photon or Z boson, which then decays into a lepton-antilepton pair, is an important channel for studying electroweak interactions and searching for new physics.

    \item Top quark pair production: The scattering process involving two top quarks being produced and subsequently decaying into multiple jets and leptons is a crucial probe of the top quark's properties and its role in electroweak symmetry breaking.

    \item Higgs boson production and decay: The scattering process involving two Higgs bosons decaying into two Z bosons is a significant channel for studying the properties of the Higgs boson at the LHC.
    
    \item Top quark decay: The scattering process involving two top quarks decaying into two W bosons is a key channel for measuring the top quark mass and its coupling to the Higgs boson.

    \item Vector boson scattering: The scattering process involving two W or Z bosons scattering off each other is a pivotal test of the electroweak theory and has the potential to reveal new physics beyond the Standard Model.
\end{itemize}
The EFT operators of other scattering processes such as the annihilation of dark matter can be calculated using the program \url{https://github.com/zizhengzhou/MassiveAmplitude}, here we only give the operators of the above-mentioned processes.

\input{result.tex}



\end{document}